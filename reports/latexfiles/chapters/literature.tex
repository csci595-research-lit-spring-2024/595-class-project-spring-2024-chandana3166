\chapter{Literature Review}
\label{ch:lit_rev} %Label of the chapter lit rev. The key ``ch:lit_rev'' can be used with command \ref{ch:lit_rev} to refer this Chapter.

This section explores the efficacy of machine learning (ML) in the prediction of cardiovascular disease (CVD). ML learns from data and experience through training, enabling it to be applied to various tasks based on specific algorithms. This flexibility enables ML algorithms to analyze complex datasets and predict CVD risk.

A review of existing literature is also done to investigate previously published studies in the area. This review  helps to contextualize the current findings. The literature review for this study will look at earlier research on machine learning (ML) in disease prediction, including various algorithms and their effectiveness. This understanding is crucial for developing accurate and effective predictive models for CVD.

% PLEAE CHANGE THE TITLE of this section
\section{Review of State-of-the-art} 
% Note the use of \cite{} and \citep{}
Previous studies, such as that by \cite{seema2016predictive}, have focused on predicting chronic diseases by analyzing data from historical health records. They employed various techniques, including Naïve Bayes, Decision Trees, Support Vector Machines (SVM), and Artificial Neural Networks (ANN). Through a comparative study of these classifiers, the researchers evaluated their performance in terms of accuracy. Their findings indicate that SVM achieved the highest accuracy rate overall, while Naïve Bayes showed the best performance in predicting diabetes.

\cite{shetty2016different} proposed the development of a predictive system for diagnosing heart disease using patient medical datasets. They considered 13 risk factors as input attributes for building the system. The data from the dataset was analyzed, and processes such as data cleaning and data integration were performed.
\

\cite{pal2021prediction}In this study, the researchers implemented the random forest data mining algorithm to predict heart disease. Their experimental results showed a sensitivity of 90.6, specificity of 82.7, and an overall accuracy of 86.9 for heart disease prediction. The proposed system achieved a classification accuracy of 86.9 and a diagnosis rate of 93.3 using the random forest algorithm. The researchers suggest that the system could also be used for predicting other diseases by applying different machine learning algorithms such as Naïve Bayes, decision tree, K-NN, linear regression, and fuzzy logic to improve accuracy. They also propose the use of cloud computing technology to manage large volumes of patient data.

\section{Example of ``risk'' of unintentional plagiarism}
Using other sources, ideas, and material always bring with it a risk of unintentional plagiarism. 

\noindent
\textbf{\color{red}MUST}: do read the university guidelines on the definition of plagiarism as well as the guidelines on how to avoid plagiarism~\citep{uor_plagiarism}.




% A possible section of you chapter
\section{Critique of the review} % Use this section title or choose a betterone
Describe your main findings and evaluation of the literature. ~\\

% Pleae use this section
\section{Summary} 
Write a summary of this chapter~\\
