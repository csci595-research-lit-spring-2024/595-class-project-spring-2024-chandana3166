\chapter{Reflection}
I gained a great deal of knowledge over the project that went beyond simply picking up new technical abilities. I was able to hone my critical thinking skills, problem-solving techniques, and research methodology.

One of the most important aspects of my learning process was recognizing and addressing difficulties. I gained the ability to deconstruct difficult issues into manageable chunks that I could work on methodically. With this iterative process, I was able to improve consistently and modify my plan of action as needed. I also came to understand how crucial it is to have an open mind because I frequently had to reevaluate my presumptions and consider other options in order to overcome unforeseen obstacles.

An additional beneficial component of the project was the research inquiry process. I gained knowledge on how to create precise research questions, carry out literature evaluations, and locate pertinent information sources. I was able to place my thesis in the larger framework of previous research and develop a deeper understanding of the subject matter thanks to this procedure. Additionally, I developed my ability to synthesize data from many sources, which helped me find trends, make connections, and come up with fresh ideas.

The project has been an enriching learning experience, providing me with valuable skills in programming, data analysis, and report writing. It has deepened my understanding of machine learning and artificial intelligence, highlighting both challenges and opportunities in these fields.

Despite the valuable insights gained, I encountered challenges that I was unable to fully overcome. One such challenge was optimizing the performance of certain machine learning models, despite extensive efforts in hyperparameter tuning and feature engineering. Looking back, I realize that I could have dedicated more time to exploring alternative models or ensembles of models, which might have led to better outcomes.

For future iterations of this project, I would approach it differently. I would allocate more time to experimenting with different modeling techniques and thoroughly evaluating their performance. Additionally, I would seek closer collaboration with domain experts to gain a deeper understanding of the clinical context and incorporate their insights into the model development process.

