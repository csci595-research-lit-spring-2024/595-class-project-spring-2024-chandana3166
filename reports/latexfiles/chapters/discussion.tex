\chapter{Discussion and Analysis}
\label{ch:evaluation}

The performance comparison of the SVM and Random Forest models for the classification of heart disease is evaluated and examined in this chapter.


\section{Discussion on model performance}
The results show that the Random Forest model outperformed the SVM model in terms of training and testing accuracy, precision, recall, and F1 score. Specifically, the Random Forest model achieved a training accuracy of 91.29 and a testing accuracy of 80.33, while the SVM model achieved a training accuracy of 86.31 and a testing accuracy of 77.05.

The training and testing confusion matrices for the SVM model show that it correctly classified 85 and 23 cases of the negative class (no heart disease), respectively, and 123 and 24 cases of the positive class (heart disease), respectively. However, the SVM model misclassified 21 cases of the negative class as positive and 5 cases of the positive class as negative in the training and testing datasets, respectively.

On the other hand, the training and testing confusion matrices for the Random Forest model show that it correctly classified 95 and 89 cases of the negative class, respectively, and 123 and 24 cases of the positive class, respectively. However, the Random Forest model misclassified 11 cases of the negative class as positive and 1 case of the positive class as negative in the training dataset, and 5 cases of the negative class as positive in the testing dataset.



\section{Significance of the findings}
The key finding of this study is that the Random Forest model outperformed the SVM model in heart disease classification, achieving higher accuracy, precision, recall, and F1 score values for both training and testing datasets. This suggests that the Random Forest model is more effective in capturing the complex relationships between the features in the heart disease dataset.

The significance of this finding lies in the potential for Random Forest models to improve the accuracy of heart disease diagnosis, which can lead to better patient outcomes. The results of this study demonstrate that Random Forest models can achieve high accuracy rates in heart disease classification, even with imbalanced datasets. This is important because imbalanced datasets are common in medical applications, where the prevalence of certain diseases may be low.

\section{Limitations}  
There are several key limitations to this study that should be taken into account when interpreting the findings. First, the study used a relatively small dataset of 303 patients, which may limit the generalizability of the results to larger and more diverse populations. Second, the study did not consider the impact of missing data or data imputation methods, which may affect the performance of the models. Third, the study used a fixed set of hyperparameters for the Random Forest model, which may not be optimal for other datasets or applications.


To address these limitations, future studies could consider using larger and more diverse datasets. Additionally, future studies could consider comparing the performance of the Random Forest model to other machine learning models, such as neural networks or gradient boosting models, to further evaluate the strengths and limitations of each approach.
\section{Summary}
In Summary, this research contributes significantly to the heart disease classification field by showing the potential of machine learning models to enhance the accuracy of heart disease diagnosis. Nevertheless, the research should be continued to fix the shortcomings of the study and to examine the consequences and possible enhancements of the results.
