\chapter{Conclusions and Future Work}
\label{ch:con}
\section{Conclusions}
After conducting a performance evaluation of the Support Vector Machine (SVM) and Random Forest models for predicting heart disease, several important conclusions can be drawn.

Both models demonstrated competitive performance in classifying heart disease, with the Random Forest model slightly outperforming the SVM model. The Random Forest model achieved higher accuracy, precision, recall, and F1 score on both the training and testing sets compared to the SVM model. This suggests that the Random Forest model may be more effective in predicting heart disease based on clinical characteristics.

Models demonstrated the ability to generalize well to unseen data, as evidenced by their performance on the testing set. This indicates that the models are not overfitting to the training data and can effectively classify heart disease in new patients. This is an important consideration for the clinical applicability of these models, as they must be able to accurately predict heart disease in patients with different clinical characteristics.

 Random Forest model provides information about feature importance, which can help identify the most relevant clinical characteristics for predicting heart disease. This analysis can provide valuable insights for medical practitioners, as it can help them understand which clinical characteristics are most strongly associated with heart disease.

Results suggest that machine learning models, particularly Random Forest, can be valuable tools in assisting medical professionals in diagnosing heart disease. By leveraging patient data, these models can provide additional support in decision-making processes, ultimately leading to better patient outcomes.


In conclusion, this study demonstrates the potential of machine learning models in predicting heart disease based on clinical characteristics. These models can serve as valuable decision support tools in healthcare settings, aiding in early detection and management of heart disease. By leveraging patient data, machine learning models can help medical professionals make more informed decisions, leading to better patient outcomes.

\section{Future work} \begin{itemize} \item \textbf{Model Tuning:} Although the SVM and Random Forest models yielded promising results, refining their hyperparameters and feature selection could potentially enhance their performance. Techniques such as genetic algorithms or Bayesian optimization can be utilized to fine-tune the models for better accuracy and generalization.

\item \textbf{Combining Models:} Investigating ensemble methods, such as stacking or boosting, could be advantageous. By merging multiple models, the aim is to capitalize on the strengths of each base model and potentially achieve superior predictive performance.

\item \textbf{Feature Creation:} Exploring more sophisticated feature engineering techniques, such as polynomial features, interaction terms, or domain-specific transformations, could lead to the discovery of more informative features for heart disease prediction.

\item \textbf{Data Expansion:} As the dataset used in this project is relatively small, expanding the dataset through techniques like synthetic data generation or oversampling of minority classes could help improve model performance, particularly in handling imbalanced datasets.

\item \textbf{Real-World Validation:} Validating the models on external datasets from different sources or populations could strengthen the models' generalizability and provide more robust results.

\item \textbf{Clinical Collaboration:} Collaborating with healthcare professionals to incorporate additional clinical insights and domain knowledge into the model development process could further enhance the models' relevance and applicability in real-world clinical settings.

\item \textbf{Model Interpretation:} Improving the interpretability of the models by using techniques such as SHAP (SHapley Additive exPlanations) values or LIME (Local Interpretable Model-agnostic Explanations) could help in understanding the factors influencing the model's predictions, making them more transparent and trustworthy for clinical use. \end{itemize}

By addressing these future work areas, the project can progress the field of heart disease prediction using machine learning, ultimately contributing to improved diagnostic accuracy and patient care.
