%Two resources useful for abstract writing.
% Guidance of how to write an abstract/summary provided by Nature: https://cbs.umn.edu/sites/cbs.umn.edu/files/public/downloads/Annotated_Nature_abstract.pdf %https://writingcenter.gmu.edu/guides/writing-an-abstract
\chapter*{\center \Large  Abstract}
%%%%%%%%%%%%%%%%%%%%%%%%%%%%%%%%%%%%%%
% Replace all text with your text
%%%%%%%%%%%%%%%%%%%%%%%%%%%%%%%%%%%

In recent times, the global rise in cardiovascular diseases has become increasingly prevalent, influenced by evolving lifestyles and societal factors. Emphasizing the need for timely detection and ongoing monitoring, particularly in regions with limited medical resources. Utilizing a public health dataset on patient heart health, including information from medical procedures and ongoing patient monitoring, this research uniquely centers on the comparative analysis of SVM and Random Forests. Focused on these two algorithms, this research aligns with the evolving landscape of machine learning in healthcare, presenting a concentrated perspective on their potential contributions. The methodology involves training SVM and Random Forest models on the dataset, evaluating their performance using key accuracy metrics such as the confusion matrix, Accuracy, precision, and F1 score. The study anticipates achieving comparable accuracy between the models but aims to determine their relative strengths in precision, recall, and F1 scores. This research aims to provide insights into which algorithm may be better suited for addressing the challenges associated with cardiovascular health monitoring, taking into consideration all parameters assessed in the research conclusion.
\\

\noindent % Provide your key words
\\

\textbf{Keywords:} Machine Learning, Random Forest, Support Vector Machine, Cardiovascular disease , Healthcare

\vfill
\noindent


