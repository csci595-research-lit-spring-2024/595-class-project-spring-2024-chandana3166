\chapter{Introduction}
\label{ch:into} % This how you label a chapter and the key (e.g., ch:into) will be used to refer this chapter ``Introduction'' later in the report. 
% the key ``ch:into'' can be used with command \ref{ch:intor} to refere this Chapter.

Cardiovascular diseases represent a formidable global health challenge, with their prevalence 
escalating and ranking among the primary causes of morbidity and mortality. The pressing 
concern is the need for robust predictive models to address the increasing burden of heart 
diseases, enabling early detection and effective risk mitigation strategies.
\\


  

%%%%%%%%%%%%%%%%%%%%%%%%%%%%%%%%%%%%%%%%%%%%%%%%%%%%%%%%%%%%%%%%%%%%%%%%%%%%%%%%%%%
\section{Background}
\label{sec:into_back}

This research project delves into the development and 
evaluation of predictive models for heart disease using machine learning algorithms, 
specifically focusing on Support Vector Machines (SVM) and Random Forests(RF). The scope encompasses 
a comprehensive analysis of these algorithms, exploring their capabilities and limitations in 
accurately predicting the risk of cardiovascular events. The context of the project revolves 
around leveraging a diverse dataset derived from various medical procedures and continuous 
patient monitoring to enhance our understanding of heart disease prediction.
\\

The background of this study lies in the evolving landscape of lifestyle, dietary habits, and 
healthcare dynamics that contribute to the increasing prevalence of cardiovascular diseases. 
The significance of early detection and continuous monitoring underscores the importance of 
advanced \cite{toma2023predictive}predictive modeling techniques. Against this backdrop, the research aims to contribute 
to the field of cardiovascular health by providing insights into the efficacy of SVM and Random 
Forest algorithms.

\\



In summary, the investigated problem centers on the escalating prevalence of cardiovascular diseases, and the project's scope involves the development and evaluation of predictive models using SVM and Random Forest. The background highlights the contextual relevance of advanced predictive modeling in addressing the challenges posed by heart diseases in the contemporary healthcare landscape.


%%%%%%%%%%%%%%%%%%%%%%%%%%%%%%%%%%%%%%%%%%%%%%%%%%%%%%%%%%%%%%%%%%%%%%%%%%%%%%%%%%%
\section{Problem statement}
\label{sec:intro_prob_art}

The research question guiding this study is: "How do Support Vector Machines (SVM) and Random Forest algorithms differ in terms of accuracy, efficiency, and interpretability when predicting the risk of heart disease?"

The prevalence of cardiovascular diseases is increasing globally, necessitating accurate and efficient predictive models for early detection and intervention. However, selecting the most suitable algorithm for this task poses a challenge. This research aims to compare and examine the differential performance of SVM and Random Forests in predicting the risk of heart disease. By leveraging real-world data\cite{lapp-heart-disease-dataset-1988} on patient heart health, the study seeks to uncover the unique strengths and limitations of each algorithm. Through this investigation, the research aims to provide insights into selecting appropriate machine learning algorithms to enhance cardiovascular health monitoring and decision-making in clinical practice.



%%%%%%%%%%%%%%%%%%%%%%%%%%%%%%%%%%%%%%%%%%%%%%%%%%%%%%%%%%%%%%%%%%%%%%%%%%%%%%%%%%%
\section{Aims and objectives}
\label{sec:intro_aims_obj}




This research project's main goal is to evaluate and contrast the effectiveness of Random Forest and Support Vector Machines (SVM) algorithms in relation to risk assessments for heart disease. The main objective is to advance predictive modeling methods for accurate assessment and early identification of cardiac disease.

\


The Kaggle online dataset resource website is the source of the heart disease dataset that will be analyzed for this research project. Utilize Random Forest and Support Vector Machine (SVM) algorithms to classify data related to heart illness, investigating the potential of each method.Examine and contrast the SVM and Random Forest prediction performances on the chosen datasets related to heart disease in order to pinpoint their advantages and disadvantages.

\

Methodically adjust multiple variables for both SVM and Random Forest models to investigate differences in classification results. Choose the SVM and Random Forest parameter setups that produce the most accurate and dependable classification results.
Finding the best parameter values for the best prediction results is the key objective.




%%%%%%%%%%%%%%%%%%%%%%%%%%%%%%%%%%%%%%%%%%%%%%%%%%%%%%%%%%%%%%%%%%%%%%%%%%%%%%%%%%%
\section{Solution approach}
\label{sec:intro_sol} % label of Org section
Briefly describe the solution approach and the methodology applied in solving the set aims and objectives.

Depending on the project, you may like to alter the ``heading'' of this section. Check with you supervisor. Also, check what subsection or any other section that can be added in or removed from this template.

\subsection{A subsection 1}
\label{sec:intro_some_sub1}
You may or may not need subsections here. Depending on your project's needs, add two or more subsection(s). A section takes at least two subsections. 

\subsection{A subsection 2}
\label{sec:intro_some_sub2}
Depending on your project's needs, add more section(s) and subsection(s).

\subsubsection{A subsection 1 of a subsection}
\label{sec:intro_some_subsub1}
The command \textbackslash subsubsection\{\} creates a paragraph heading in \LaTeX.

\subsubsection{A subsection 2 of a subsection}
\label{sec:intro_some_subsub2}
Write your text here...

%%%%%%%%%%%%%%%%%%%%%%%%%%%%%%%%%%%%%%%%%%%%%%%%%%%%%%%%%%%%%%%%%%%%%%%%%%%%%%%%%%%
\section{Summary of contributions and achievements} %  use this section 
\label{sec:intro_sum_results} % label of summary of results



%%%%%%%%%%%%%%%%%%%%%%%%%%%%%%%%%%%%%%%%%%%%%%%%%%%%%%%%%%%%%%%%%%%%%%%%%%%%%%%%%%%


