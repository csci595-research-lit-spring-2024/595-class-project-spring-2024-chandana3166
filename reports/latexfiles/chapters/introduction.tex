\chapter{Introduction}
\label{ch:into} % This how you label a chapter and the key (e.g., ch:into) will be used to refer this chapter ``Introduction'' later in the report. 
% the key ``ch:into'' can be used with command \ref{ch:intor} to refere this Chapter.

Cardiovascular diseases represent a formidable global health challenge, with their prevalence 
escalating and ranking among the primary causes of morbidity and mortality. The pressing 
concern is the need for robust predictive models to address the increasing burden of heart 
diseases, enabling early detection and effective risk mitigation strategies.
\\


  

%%%%%%%%%%%%%%%%%%%%%%%%%%%%%%%%%%%%%%%%%%%%%%%%%%%%%%%%%%%%%%%%%%%%%%%%%%%%%%%%%%%
\section{Background}
\label{sec:into_back}

This research project delves into the development and 
evaluation of predictive models for heart disease using machine learning algorithms, 
specifically focusing on Support Vector Machines (SVM) and Random Forests(RF). The scope encompasses 
a comprehensive analysis of these algorithms, exploring their capabilities and limitations in 
accurately predicting the risk of cardiovascular events. The context of the project revolves 
around leveraging a diverse dataset derived from various medical procedures and continuous 
patient monitoring to enhance our understanding of heart disease prediction.
\\

The background of this study lies in the evolving landscape of lifestyle, dietary habits, and 
healthcare dynamics that contribute to the increasing prevalence of cardiovascular diseases. 
The significance of early detection and continuous monitoring underscores the importance of 
advanced \cite{toma2023predictive}predictive modeling techniques. Against this backdrop, the research aims to contribute 
to the field of cardiovascular health by providing insights into the efficacy of SVM and Random 
Forest algorithms.

In \cite{kumari2023comparative}  a comparative study on classification methods namely
Ripper, Decision Tree, Artificial neural networks and Support
Vector Machine are analyzed on cardiovascular disease
dataset.

In \cite{yanwei2007combination}it is establishes that a number of factors have been
shown to increase the risk of developing heart disease. Some
of these family history, high levels of LDL bad cholesterol,
Family history of cardiovascular disease, High levels of LDL
(bad) cholesterol, Low level of HDL (good) cholesterol,
Hypertension, High fat diet, Lack of regular exercise,
Obesity.

\\



In summary, the investigated problem centers on the escalating prevalence of cardiovascular diseases, and the project's scope involves the development and evaluation of predictive models using SVM and Random Forest. The background highlights the contextual relevance of advanced predictive modeling in addressing the challenges posed by heart diseases in the contemporary healthcare landscape.


%%%%%%%%%%%%%%%%%%%%%%%%%%%%%%%%%%%%%%%%%%%%%%%%%%%%%%%%%%%%%%%%%%%%%%%%%%%%%%%%%%%
\section{Problem statement}
\label{sec:intro_prob_art}

The research question guiding this study is: "How do Support Vector Machines (SVM) and Random Forest algorithms differ in terms of accuracy, efficiency, and interpretability when predicting the risk of heart disease?"

The prevalence of cardiovascular diseases is increasing globally, necessitating accurate and efficient predictive models for early detection and intervention. However, selecting the most suitable algorithm for this task poses a challenge. This research aims to compare and examine the differential performance of SVM and Random Forests in predicting the risk of heart disease. By leveraging real-world data\cite{lapp-heart-disease-dataset-1988} on patient heart health, the study seeks to uncover the unique strengths and limitations of each algorithm. Through this investigation, the research aims to provide insights into selecting appropriate machine learning algorithms to enhance cardiovascular health monitoring and decision-making in clinical practice.



%%%%%%%%%%%%%%%%%%%%%%%%%%%%%%%%%%%%%%%%%%%%%%%%%%%%%%%%%%%%%%%%%%%%%%%%%%%%%%%%%%%
\section{Aims and objectives}
\label{sec:intro_aims_obj}




This research project's main goal is to evaluate and contrast the effectiveness of Random Forest and Support Vector Machines (SVM) algorithms in relation to risk assessments for heart disease. The main objective is to advance predictive modeling methods for accurate assessment and early identification of cardiac disease.

\


\begin{itemize}
  \item Analyze and clean the Kaggle heart disease dataset, preparing it for building predictive models.
  \item Develop a Support Vector Machine (SVM) classifier and train it on the preprocessed heart disease data.
  \item Build a Random Forest (RF) classifier and train it on the same preprocessed data.
  \item Conduct parameter tuning to maximize the performance of both models, focusing on recall, precision, and/or F1 scores.
  \item Compare the recall, precision, and  accuracy of the resulting SVM and RF models.
  \item Evaluate and interpret the performance of SVM and RF models in predicting heart disease risk.
  \item Identify the strengths and weaknesses of each model in the context of cardiovascular health monitoring.
  \item Draw conclusions and provide recommendations for selecting the most suitable machine learning algorithm for heart disease risk analysis.
\end{itemize}



%%%%%%%%%%%%%%%%%%%%%%%%%%%%%%%%%%%%%%%%%%%%%%%%%%%%%%%%%%%%%%%%%%%%%%%%%%%%%%%%%%%
\section{Solution approach}
\label{sec:intro_sol} % label of Org section
\textbf{Data Preparation:} Prepare the Kaggle heart disease dataset for the construction of predictive models by analyzing and cleaning it.
\\
\textbf{Model Development:} Using the preprocessed cardiac disease data, create an SVM classifier and train it. Using the same preprocessed data, create and train a Random Forest (RF) classifier.
\\
\textbf{Model Assessment:} Optimize both models' performance by fine-tuning their parameters with an emphasis on recall, precision, and/or F1 scores.

Print the assessment measures for each of the two models:
\begin{itemize}
    \item Metrics for the SVM Model: F1 Score, Accuracy, Precision, Recall.
    \item Metrics for the Random Forest Model: F1 Score, Accuracy, Precision, Recall.
\end{itemize}

Print the two models' confusion matrices.
\\
\textbf{Model Comparison:} Based on the evaluation metrics and confusion matrices, compare the SVM and Random Forest models' performances. Analyze and assess how well the RF and SVM models predict the risk of heart disease. Determine each model's advantages and disadvantages in relation to cardiovascular health monitoring.
\\
\textbf{Concluding remarks and suggestions:}
\begin{itemize}
    \item Make inferences from the performance comparison.
    \item Make suggestions on which machine learning algorithm would be best for analyzing the risk of heart disease.
\end{itemize}

The measures used to answer the research topic of comparing and contrasting the efficiency of Random Forest and SVM algorithms for heart disease risk assessments are described in this solution methodology. Data preparation, model construction, assessment, comparison, and conclusion are all included, giving your research a thorough approach.



%%%%%%%%%%%%%%%%%%%%%%%%%%%%%%%%%%%%%%%%%%%%%%%%%%%%%%%%%%%%%%%%%%%%%%%%%%%%%%%%%%%
\section{Summary of contributions and achievements} %  use this section 
\label{sec:intro_sum_results} % label of summary of results



%%%%%%%%%%%%%%%%%%%%%%%%%%%%%%%%%%%%%%%%%%%%%%%%%%%%%%%%%%%%%%%%%%%%%%%%%%%%%%%%%%%


